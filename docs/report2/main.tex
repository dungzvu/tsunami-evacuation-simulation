\documentclass[11pt,a4paper]{article}
\usepackage[utf8]{inputenc}
\usepackage[a4paper,vmargin={17mm,20mm},hmargin={20mm,20mm}]{geometry}
\usepackage{amsmath}
\usepackage{amssymb}
\usepackage{mathtools}
\usepackage{gensymb}
\usepackage{enumitem}
\usepackage{graphicx}
\usepackage{scrextend}
\usepackage{blindtext}
\usepackage{caption}
\usepackage{url}
\usepackage{subcaption}
\usepackage{circuitikz}
\usepackage{listings}
\usepackage{color}
\usepackage[defaultsans,scale=0.9]{opensans}
 
\definecolor{codegreen}{rgb}{0,0.6,0}
\definecolor{codegray}{rgb}{0.5,0.5,0.5}
\definecolor{codepurple}{rgb}{0.58,0,0.82}
\definecolor{backcolour}{rgb}{0.95,0.95,0.92}
 
\lstdefinestyle{mystyle}{
    backgroundcolor=\color{backcolour},   
    commentstyle=\color{codegreen},
    keywordstyle=\color{magenta},
    numberstyle=\tiny\color{codegray},
    stringstyle=\color{codepurple},
    basicstyle=\footnotesize,
    breakatwhitespace=false,         
    breaklines=true,                 
    captionpos=b,                    
    keepspaces=true,                 
    numbers=left,                    
    numbersep=5pt,                  
    showspaces=false,                
    showstringspaces=false,
    showtabs=false,                  
    tabsize=2
}
 
\lstset{style=mystyle}
 

% \DeclarePairedDelimiter\floor{\lfloor}{\rfloor}
% \title{Final Project Report}
% \date{\today}
% \author{By Akashdeep Deb
% \sloppy
\begin{document}
\title{\textbf{Modeling and Simulation - Prof. Alexis Drogoul\\Project Topic 4: Evacuation}}

\author{Vu Trung Dung}
\date{\today}
\maketitle

%this is the tenth question

\begin{abstract}

\begin{addmargin}[3em]{1em}
\centering
Populations are increasingly vulnerable to disastrous natural or technological events, as demographic and urban growth lead to greater exposures of goods and people. Hanoi, for example, is particularly hard hit by flooding. Some districts on the banks of the Red River are also threatened by potential dike breaching. In the event of a levee failure, it is important to be able to evacuate the population living in these areas before the water arrives.
\end{addmargin}

\end{abstract}

\section{Introduction}
%starting off with the answer
\subsection{Problem Statement}
\begin{large}
The goal of this project is to build an agent-based model of people evacuation, to have a better understanding of the evacuation process and to be able to test different evacuation strategies. The question to be answered is: \textbf{How to better manage the pedestrian evacuation of a population on a beach in a tsunami context?} \\
The goal is to optimize the evacuation process according to the following criteria:
\begin{itemize}
    \item Minimize the time people spend wandering on the roads to inform others about the threat.
    \item Maximize the number of people informed and evacuated.
    \item Minimize the total evacuation time as well.
\end{itemize}

\subsection{Project Extensions}

The project will be divided into two parts: the first part will be to implement the basic model described above, and the second part will be to implement extensions to the model.

In this model, flooding will not be modeled by itself, just the behavior of residents in the face of the threat. People will only evacuate if they have been informed of the imminent risk of flooding. At the start of the simulation, we assume that all residents are located in their own homes and only 10\% of the population (randomly chosen) will be aware of this information. Once informed, people will evacuate to the shelter (the largest building in the area). A person observing someone evacuating (at a distance of less than 10m) will have a probability of 0.1 of evacuating in turn.

\begin{itemize}
\item[1] \textbf{Extension 1}: Only 10\% of the population go directly to the shelter when informed of the risk. The other will move to random buildings to inform other residents and search for the shelter.
\item[2] \textbf{Extensions 2}: Implement the multiple modalities of evacuation (car, motocycle, walking) and the possibility of blocking roads.
\item[3] \textbf{Extensions 3}: Experiment with different evacuation strategies of informing the threat to the 10\% of the population: those furthest from shelter, those closest to the shelter, or randomly selected.
\end{itemize}

\end{large} 
\end{document}
