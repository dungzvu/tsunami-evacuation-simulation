\documentclass{beamer}
\usetheme{uic}
\usepackage{amsfonts,amsmath,oldgerm,algorithm,algpseudocode}
\usepackage[font=small,labelfont=bf]{caption} % Required for specifying captions to tables and figures

\newcommand{\hrefcol}[2]{\textcolor{uihteal}{\href{#1}{#2}}}
\newcommand{\testcolor}[1]{\colorbox{#1}{\textcolor{#1}{test}}~\texttt{#1}}

% Please see Section 18.1 of Beamer User Guide for all the options \usefonttheme provides
\usefonttheme[onlymath]{serif}
% \usefonttheme{serif} % use this if you would like Serif font throughout (and not just for math)


\titlebackground*{assets/uic_lockup_blue.pdf}

% NOTE 1: The asterisk splits the background image. This option is good
% for logo-based backgrounds. If you use an image based background
% it's recommended to not split it:

% \titlebackground{assets/uic_seo.jpg}

% NOTE 2: If you use a title background that does not have a logo, you might
% want to enable logo in the top left.
% To do that, simply comment out this line
\themecolor{lightnologo}

\title{Get started with the UIC Beamer Theme}
\subtitle{Using \LaTeX\ to prepare slides}
% This can be adjusted accordingly for longer titles
\setlength{\titleboxwidth}{0.45\textwidth}

\author{\href{mailto:umunee2@uic.edu}{Usama Muneeb}}
\date{\today}

\begin{document}
\maketitle
\themecolor{lightnologo} % reverts to a logo based theme (if you disabled it for title page)

% enabled after title page creation (i.e. after \maketitle)
\footlinecolor{uicblue}


\begin{frame}[fragile]{Beamer for UIC presentations}
\framesubtitle{A quick start}
If you would like \LaTeX\ in your presentation, Beamer is a great way to go!
\begin{itemize}
\item Beamer has a detailed
\hrefcol{https://www.ctan.org/tex-archive/macros/latex/contrib/beamer/doc/beameruserguide.pdf}{user
 manual}, but we will go over the most common features.
\item This template is designed for a 16:9 aspect ratio, which is the default in PowerPoint and the most common amongst projectors.
\item The most common of all slide types involve bulleted points, like these. \pause
\begin{itemize}
\item Placing \verb|\pause| after items will allow you to sequentially unroll points.
\item Changing \verb|\begin{itemize}| to \verb|\begin{itemize}[<+->]| achieves a similar effect, although \verb|\pause| gives finer control on the unrolling.
\end{itemize}
\end{itemize}
\end{frame}


% \backmatter
\end{document}
