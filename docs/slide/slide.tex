\documentclass{beamer}
\usetheme{uic}
\usepackage{amsfonts,amsmath,oldgerm,algorithm,algpseudocode}
\usepackage[font=small,labelfont=bf]{caption} % Required for specifying captions to tables and figures

\newcommand{\hrefcol}[2]{\textcolor{uihteal}{\href{#1}{#2}}}
\newcommand{\testcolor}[1]{\colorbox{#1}{\textcolor{#1}{test}}~\texttt{#1}}

% Please see Section 18.1 of Beamer User Guide for all the options \usefonttheme provides
\usefonttheme[onlymath]{serif}
% \usefonttheme{serif} % use this if you would like Serif font throughout (and not just for math)


\titlebackground*{assets/usth_lockup_blue.png}

% NOTE 1: The asterisk splits the background image. This option is good
% for logo-based backgrounds. If you use an image based background
% it's recommended to not split it:

% \titlebackground{assets/uic_seo.jpg}

% NOTE 2: If you use a title background that does not have a logo, you might
% want to enable logo in the top left.
% To do that, simply comment out this line
\themecolor{lightnologo}

\title{Modeling and simulation of complex systems}
\subtitle{Project 4: Evacuation}
% This can be adjusted accordingly for longer titles
\setlength{\titleboxwidth}{0.45\textwidth}

\author{\href{mailto:dungvt2440071@usth.edu.vn}{Vu Trung Dung}}
\date{\today}

\begin{document}
\maketitle
\themecolor{lightnologo} % reverts to a logo based theme (if you disabled it for title page)

% enabled after title page creation (i.e. after \maketitle)
\footlinecolor{uicblue}


% ------------------------------------------------
% Question
% ------------------------------------------------
\begin{frame}[fragile]{Question}
% \framesubtitle{}
\textbf{How to better manage the pedestrian evacuation of a population on a beach
in a tsunami context?}

\begin{itemize}
\item flooding will not be modeled by itself
\item just \textbf{the behavior of residents} in the face of the threat.
\end{itemize}
\end{frame}


% ------------------------------------------------
% Modeling
% ------------------------------------------------
\begin{frame}[fragile]{Modeling and Simulation}
\framesubtitle{Situation}
\begin{itemize}
\item People will only evacuate if they have been informed of the imminent risk of flooding. 
\begin{itemize}
    \item We assume that only \textbf{10\% of the population is informed at the beginning of the simulation}.
    \item A person observing someone evacuating
    (at a distance of less than 10m) will have a probability of 0.1 of evacuating in turn.
\end{itemize}
\item \textbf{Not all residents know where to evacuate} and only 10\% will go directly to the shelter.
\item People have multiple modalities of evacuation: they can evacuate by car, by bike, or on foot.
\end{itemize}
\end{frame}

\begin{frame}[fragile]{Modeling and Simulation}
\framesubtitle{Situation - My extension}
\begin{itemize}
\item \textbf{The knowledge of the evacuation shelter can be transfered across people}.
\begin{itemize}
    \item The knowledge of the evacuation shelter can be shared with 2 people at 10\% probability when they meet each other on the road while evacuating.
    \item When the knowledge is shared, 2 people have 10\% probability to change their evacuation to the closest shelter they're heading to.
\end{itemize}
\end{itemize}
\end{frame}

\begin{frame}[fragile]{Modeling and Simulation}
\framesubtitle{Strategies to aware of flooding}

Different strategies of aware of flooding to the 10\% of the population:
\begin{itemize}
    \item random.
    \item furthest from the shelter.
    \item closest to the shelter.
\end{itemize}

\textbf{Find the most effective} of these strategies in terms of:
\begin{itemize}
    \item number of evacuees.
    \item evacuation time.
    \item time for the total evacuation/time spent on the roads.
\end{itemize}
\end{frame}


\begin{frame}[fragile]{Modeling and Simulation}
\framesubtitle{Parameters}

\begin{itemize}
    \item initial population.
    \item the alert time before the flooding.
\end{itemize}

\end{frame}


% ------------------------------------------------
% GIS DATA
% ------------------------------------------------
\begin{frame}[fragile]{GIS Data}
% \framesubtitle{The }
GIS data of the city of Hanoi, Vietnam.

% TODO: insert the map image

\end{frame}


% ------------------------------------------------
% Implementation - Extensions
% ------------------------------------------------
\begin{frame}[fragile]{Implementation (GAMA)}
\framesubtitle{Extensions}

\begin{itemize}
    \item \textbf{Extensions 0}: GIS map, population, evacuation shelter, roads, flooding simulation, etc.
    \item \textbf{Extensions 1}: The evacuating behavior of the population.
    \item \textbf{Extensions 2}: Multimobility of population (car, bike, foot).
    \item \textbf{My Extensions}: The knowledge of the evacuation shelter can be transfered across people.
    \item \textbf{Extensions 3}: Experiment and analyze the effectiveness of different strategies of aware of flooding.
\end{itemize}

\end{frame}


% ------------------------------------------------
% Implementation - Extensions 0
% ------------------------------------------------
\begin{frame}[fragile]{Implementation: Extensions 0}
\framesubtitle{The Map}


\end{frame}


% ------------------------------------------------
% Implementation - Extensions 0
% ------------------------------------------------
\begin{frame}[fragile]{Implementation: Extensions 0}
\framesubtitle{Species}

\begin{itemize}
    \item \textbf{People}: the inhabitants of the city.
    \item \textbf{Evacuation Shelter}: the shelter for evacuation.
    \item \textbf{Road}: the road for evacuation.
    \item \textbf{Building}: the building in the city.
    \item \textbf{Flooding}: the flooding area.
\end{itemize}

\end{frame}

% ------------------------------------------------
% Implementation - Extensions 0
% ------------------------------------------------
\begin{frame}[fragile]{Implementation: Extensions 0}
\framesubtitle{Species - Flooding}



\end{frame}

% ------------------------------------------------
% Implementation - Extensions 1
% ------------------------------------------------
\begin{frame}[fragile]{Implementation: Extensions 1}
\framesubtitle{Species - People behavior}

% Random 10\% of the population is informed at the beginning of the simulation.
% A person observing someone evacuating (at a distance of less than 10m) will have a probability of 0.1 of evacuating in turn.
% Not all residents know where to evacuate and only 10\% will go directly to the shelter.

\end{frame}


% ------------------------------------------------
% Implementation - Extensions 2
% ------------------------------------------------
\begin{frame}[fragile]{Implementation: Extensions 2}
\framesubtitle{Multimobility of population}

% People have multiple modalities of evacuation: they can evacuate by car, by bike, or on foot.
% Traffic jam, speed, etc.

\end{frame}

% ------------------------------------------------
% Implementation - Extensions 2
% ------------------------------------------------
\begin{frame}[fragile]{Implementation: My Extensions}
\framesubtitle{Transfer knowledge of evacuation shelter}
    
% The knowledge of the evacuation shelter can be shared with 2 people at 10\% probability when they meet each other on the road while evacuating.
% When the knowledge is shared, 2 people have 10\% probability to change their evacuation to the closest shelter they're heading to.
% Expose as a new parameter.
    
\end{frame}

% ------------------------------------------------
% Implementation - Extensions 3 - Random
% ------------------------------------------------
\begin{frame}[fragile]{Implementation: Extensions 3}
\framesubtitle{Trategies: Random}
    

\end{frame}

% ------------------------------------------------
% Implementation - Extensions 3 - Furthest
% ------------------------------------------------
\begin{frame}[fragile]{Implementation: Extensions 3}
\framesubtitle{Trategies: Furthest}
    

\end{frame}

% ------------------------------------------------
% Implementation - Extensions 3 - Closet
% ------------------------------------------------
\begin{frame}[fragile]{Implementation: Extensions 3}
\framesubtitle{Trategies: Closet}
    

\end{frame}

% ------------------------------------------------
% Experiment - Comparison
% ------------------------------------------------
\begin{frame}[fragile]{Experiment}
\framesubtitle{Comparison}

% Setup the randomness.
% A grid of experiments to compare the effectiveness of different strategies of aware of flooding. Y is the strategy, X is the parameter (initial population, alert time before the flooding, transfer knowlege on/off).
% Find the sensitivity of the parameters.

\end{frame}

% ------------------------------------------------
% Experiment - Batch exploration
% ------------------------------------------------
\begin{frame}[fragile]{Experiment}
\framesubtitle{Batch exploration}

% Batch exploration to find the most effective strategy.

\end{frame}

\backmatter
\end{document}
